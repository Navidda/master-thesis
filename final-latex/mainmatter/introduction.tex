\chapter{Introduction}

% Three Pages in total

% \lipsum[1-2]  % Dummy text, replace with your content

\section{Motivation}
% First start with context and background
% Then explain the problem and its importance

% How much is the cost of cybercrime?
By the increasing use of information technology in many sectors, organizations are facing more cyber threats than ever \cite{skopik_problem_2016}. The loss incurred by cybercrime is huge and is  increasing every year, increased from \$3 to \$6 trillion annually from 2015 to 2021 \cite{noauthor_2023_nodate}.
% TODO: Mention the recent cyberattacks on infrastructures such as ...

% How much are they spending on cybersecurity?
% (Instead) The difficulty of security of processes
Due to the importance of cybersecurity, organizations are spending more to protect their systems. It is estimated that the total spending on cybersecurity will exceed \$1.75 trillion from 2021-2025 \cite{freeze_global_2021}

% Intelligence
% What is cyber threat intelligence?
To protect against the threats, one should understand them first. To do so, one should collect, process, analyze, disseminate the information about the threats. It includes information about threat actors, their motivations, tactics, techniques, and procedures (TTPs), indicators of compromises (IOCs), the systems' vulnerabilities, incident response plans and mitigation strategies. This process results in Cyber Threat Intelligence (CTI). The availabity of more information sources allows an organization to collect more reliable and useful intelligence it can obtain.

% CTI sharing
It is often useful to share this intelligence externally, called CTI sharing. A reason is that the threat landscape that different organizations face could be similar due the commonalities they have, such as systems, procedures and adversaries. CTI sharing often happens as information sharing communities, where several organizations collaborate as allies by sharing information to get stronger collective defense.

% Why is it difficult to share cyber threat intelligence?
However, sharing CTI is not a trivial task. Establishing trust, achieving interoperability and automation, dealing with sensitive or classified information, infrastructure for managing external information, validation of quality of information are examples of the challenges of CTI sharing. \cite{johnson_guide_2016}

% What is Dataspaces and why it might be useful?
TODO

% It is not studied enough in the context of CCTI
To the best of our knowledge, the concept of dataspaces is not studied in the context of CCTI.

\section{Objectives}
% List main research objectives: Research questions
% How these objectives are achieved: methodology
% Why these objectives are significant


\section{Contributions}
% List main research contributions
% More results based
% Why these contributions are significant

% Contributions
% 	- Use Case analysis -> Evaluation (?) Expert Evaluation
% 		- info model
% 	- System Design -> Analytical Evaluation
% 		- info model
% 	- Prototype -> Experimental Evaluation
% 		- technology comparison

\section{Outline}
% List of chapters and their content
